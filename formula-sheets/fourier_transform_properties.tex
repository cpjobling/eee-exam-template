%% Properties of the Fourier transform

\begin{longtable}[]{@{}
  >{\raggedleft\arraybackslash}p{(\columnwidth - 8\tabcolsep) * \real{0.1000}}
  >{\raggedright\arraybackslash}p{(\columnwidth - 8\tabcolsep) * \real{0.2000}}
  >{\raggedright\arraybackslash}p{(\columnwidth - 8\tabcolsep) * \real{0.200}}
  >{\raggedright\arraybackslash}p{(\columnwidth - 8\tabcolsep) * \real{0.20}}
  >{\raggedright\arraybackslash}p{(\columnwidth - 8\tabcolsep) * \real{0.3}}@{}}
\toprule\noalign{}
\begin{minipage}[b]{\linewidth}\raggedleft
No.
\end{minipage} & \begin{minipage}[b]{\linewidth}\raggedright
\textbf{Property}
\end{minipage} & \begin{minipage}[b]{\linewidth}\raggedright
\(f(t)\)
\end{minipage} & \begin{minipage}[b]{\linewidth}\raggedright
\(F(j\omega)\)
\end{minipage} & \begin{minipage}[b]{\linewidth}\raggedright
\textbf{Remarks}
\end{minipage} \\
\midrule\noalign{}
\endhead
\bottomrule\noalign{}
\endlastfoot
1. & Linearity & \(a_1f_1(t)+a_2f_2(t)+\cdots+a_nf_n(t)\) &
\(a_1F_1(j\omega)+a_2F_2(j\omega)+\cdots+a_nF_n(j\omega)\) & Fourier
transform is a linear operator. \\
2. & Symmetry & \(2\pi f(-j\omega)\) & \(F(t)\) & \\
3. & Time and frequency scaling & \(f(\alpha t)\) &
\(\displaystyle{\frac{1}{\lvert\alpha\rvert}F\left(j\frac{\omega}{\alpha}\right)}\)
& time compression is frequency expansion and \emph{vice versa} \\
4. & Time shifting & \(\displaystyle{f(t-t_0)}\) &
\(\displaystyle{e^{-j\omega t_0}F(j\omega)}\) & A time shift corresponds
to a phase shift in frequency domain \\
5. & Frequency shifting & \(\displaystyle{e^{j\omega_0 t}f(t)}\) &
\(\displaystyle{F(j\omega-j\omega_0)}\) & Multiplying a signal by a
complex exponential results in a frequency shift. \\
6. & Time differentiation & \(\displaystyle{\frac{d^n}{dt^n}\,f(t)}\) &
\(\displaystyle{(j\omega)^nF(j\omega)}\) & \\
7. & Frequency differentiation & \(\displaystyle{(-jt)^n f(t)}\) &
\(\displaystyle{\frac{d^n}{d\omega^n}F(j\omega)}\) & \\
8. & Time integration &
\(\displaystyle{\int_{-\infty}^{t}f(\tau)d\tau}\) &
\(\displaystyle{\frac{F(j\omega)}{j\omega}+\pi F(0)\delta(\omega)}\)
& \\[3ex]
9. & Conjugation & \(\displaystyle{f^*(t)}\) &
\(\displaystyle{F^*(-j\omega)}\) & \\[2ex]
10. & Time convolution & \(\displaystyle{f_1(t)*f_2(t)}\) &
\(\displaystyle{F_1(j\omega) F_2(j\omega)}\) & Compare with Laplace
Transform \\
11. & Frequency convolution & \(\displaystyle{f_1(t)f_2(t)}\) &
\(\displaystyle{\frac{1}{2\pi}\left(F_1(j\omega)*F_2(j\omega)\right)}\) & This has
application to amplitude modulation. \\[4ex]
12. & Area under \(f(t)\) &
\(\displaystyle{\int_{-\infty}^{\infty} f(t)\,dt = F(0)}\) & & Way to
calculate DC (or average) value of a signal \\[3ex]
13. & Area under \(F(j\omega)\) &
\(\displaystyle{f(0) = \frac{1}{2\pi}\int_{-\infty}^{\infty}F(j\omega)\,d\omega}\)
& & \\[3ex]
14. & Energy-Density Spectrum &
\(\displaystyle{E_{[\omega_1,\omega_2]}:=\displaystyle{\frac{1}{2\pi}\int_{\omega_1}^{\omega_2}\lvert F(j\omega)\rvert ^2\,d\omega.}}\)
& & \\
15. & Parseval's theorem &
\(\displaystyle{\int_{-\infty}^{\infty}\lvert f(t)\rvert^2\,dt=\displaystyle{\frac{1}{2\pi}\int_{-\infty}^{\infty}\lvert F(j\omega)\rvert ^2\,d\omega.}}\)
& & Definition of RMS follows from this \\[4ex]
\end{longtable}

\endinput
